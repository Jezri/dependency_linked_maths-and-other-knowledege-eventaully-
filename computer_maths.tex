\documentclass[12pt, a4paper,oneside]{book}
\usepackage{ucs}
\usepackage[utf8]{inputenc}
\usepackage{tikz}
\usepackage{fullpage}
\usepackage{paracol}
\usepackage{amsfonts}
\usepackage{mathtools}
\usepackage{color}

\title{Binary mathematic and Boolean logic for coms and math}
\begin{document}
{\huge{Names given in brackets are not official ones please do not use them in test ect. }}
\maketitle
\part{Binary mathematics}
Binary numbers work almost exactly like normal (denary) numbers so I will just go over normal ones to get a sense of the mechanics  
\section{Denary(Dekbazo)}
\paragraph{Denary}
\begin{paragraph}
Denary is the normal number system.  To find what quantity 136 is equal to, we multiply 6 by \(10^0\) and add that to 3 time \(10^1\) finally we add 1 times \(10^2\) so the position of the number determines what power of 10 we multiply it by.    
\end{paragraph}
\section{Binary(Dubazo)}
\paragraph{Binary}
Here we multiply numbers by \(2^n\) depending on their position. 
\end{document}
