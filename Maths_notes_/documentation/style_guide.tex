\documentclass{article}
\usepackage{ucs}
\usepackage[utf8]{inputenc}
\usepackage{fullpage}
\author{Krinsky Jezri \and Krinsky Keelan}
\title{Style Guide}
\begin{document}
\maketitle
\paragraph{Any notes should be as concise and modular as possible. 
Any information which is not imperative for understanding such as metaphors, examples, and question,designed to prompt readers belong in the Extra sections}

\paragraph{The ultimate goal of this project is to eventually transfer the document from the flat file format into a dependency tree. this dependency tree format will clearly demonstrate exactly which basics must be learnt /revised before tackling a more advanced mathematical concept, ensuring that no information is presented until it can be fully understood } . 

\paragraph{Here are some strategies\ best practice to use. All these examples are actual textbooks}
\begin{itemize}
\item{Put full proof down or not where you do not yet know them}
\item{Make the strongest possible positive statement\\ \textbf{example:}\\ 
\begin{tabular}{|l|p{14cm}|} 
\hline
Adding integers & To add integers with the same sign , add their absolute values. Give the result the same sign as the integers. \\ & To add integers with different signs, subtract the lesser absolute value from the greater absolute value. Give, the result the same sign as the integer with the greater absolute value \\ \hline
\end{tabular} 
\\ \\
Reasons for strategy\footnote {word better} \\ The example is \begin{enumerate} \item{Misleading as it implies all other numbers are handled differently }\item{ Inefficient as the method would have to be repeated when for other real numbers}\item{If the wording of this explanation was improved for one type of numbers it may not be for other types of numbers, this creating the false impression that the two are subtly different} \end{enumerate}
A common error of this type is stating things about slopes (geometric rates of change, that are true for all rates of change) \footnote{Put this in a different section}}
\item{Never write an incomplete proof}\footnote{Find example}
\item{Put a margin not by each step of the proof stating which other section is used for this step of the proof}
\item{list all sections used at the end only use sections directly referenced}
\item{sections should be atomic i.e unsplitable.}
\item{For plain English use read Wydick plain English for lawyers}
\item{All symbol definitions go in the appendix so as not to distract from the actual content}
\item{Descriptive names:\\ Many science concepts refer to ideas about a concept that are not valid  \\ \textbf{Example} \\ There is nothing unnatural about numbers not part of the natural numbers, nothing illogical about irrational numbers and imaginary numbers describe very real ideas.\\ Another problem is that many scientific words is in Latin and ancient Greek and while most of us have picked up a handful of suffix prefixes and roots the subtitles of their conjugation is known to very few. I don't know this for certain but my guess is that this results in many scientific names that don't even follow the rules of Latin grammar or apply them incorrectly. \\ \textbf{Example} \\ Many people talk about computer memory in gigabytes(base ten) when the 'correct' term is gibibyte(base 2)\\
\textbf{Reason for change} \\ Yes, we can ultimately use words to mean what ever we want to, but many names used are confusing, while the conventional names must of course be learnt for exam purposes a sensible name alongside may help clarify the concept in our own minds.\\ \textbf{Solution} \\ We put new descriptive names where necessary in brackets for these names we use the simple and formal language Esperanto. We also look for ways to put in systematic nameing systems like those in chemistry \\ \textbf{Why esperanto????}\\ \begin{enumerate}
\item{It is total unnecessary for understanding the notes it just an addition for those who want}
\item{Esperanto was designed to be a very simple language and one meant for science }
\item{Unlike Latin or Greek making new words from existing roots according to accepted grammar is easily achievable}
\item{There will be no confusion between the new words and existing but misleading Latin and Greek ones}
\item{Esperanto is based on always being able to add roots together and take them apart so the new words meanings will be self evident }
\item{You would not have to learn the language properly but only in the sense you know Latin or Greek i.e a handful of important roots and how to add them together}
\item{Because the system for adding roots together is so common you actually need to learn a much smaller vocabulary}
\item{There is no risk of new Esperanto words be unpronounceable messes the language was designed to never put unpronounceable letter combinations together }
\item{Plain English would not be simpler, all scientific English words are Latin/Greek and the alternative would be to appropriate old anglo-saxon words i.e water mot for water molecule or lifelore for biology. Most of use use these words so seldom it would be like learning a new language anyway, only one far more complicated than Esperanto }
\end{enumerate}
\textbf{How to learn the basics} \\ This website contain the basic grammar rules and suffixes that you need skip the first bit \\ http://literaturo.org/HARLOW-Don/Esperanto/affixes.html\\ And here is a online dictionary that will translate Esperanto into its actual roots \\ https://lernu.net/en/vortaro  \\ 
If you want to actually talk about it here is a pronunciation guide (it is pretty simple as each letter makes only one sound)\\ http://www.alcyone.com/max/lang/esperanto/alphabet.html\\ to add diacritics use \^c, \^H .ect \\ If you can find suitable roots but can't put them together or just can think of the English phrase which should be translated put it in with a comment and someone more familiar with the Language will fix it}
\end{itemize}
\end{document}
FYI I want to use the gender reform to Esperanto whereby all roots are neutral and i\^co is added to make a word masculine. So all math concepts are inherently neutral.  
